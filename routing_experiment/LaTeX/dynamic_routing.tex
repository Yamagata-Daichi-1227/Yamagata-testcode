%--------------------------------------------------------------
%   動的経路制御の実験
%--------------------------------------------------------------

%
% 実験の動機
% RIPとは
% quaggaとは
% 実験開始
%  - NICの設定をする
%  - そもそものquaggaを使えるようにする
%  - quaggaの起動とripの起動
%  - quaggaの設定
%  - 経路表の確認と疎通確認

\section{動的経路制御の実験}
%あらまし
静的経路制御では,各端末のNICの設定を行った後,routeコマンドを用いて手動で経路制御表の作成を行った.
しかしながら,この方法ではネットワーク構成の一部を変化させる際に大幅な改変を行うことになる.
そこで,本実験では自動的にルーティングテーブルを更新するプロコトルとして有名なRIPによる動的経路制御を行う.

\subsection{RIPプロトコル}
本実験で使用するルーティングプロトコルであるRIPについて簡単に説明する.

RIP(Routing Information Protocol)は,UDP/IP通信用のダイナミックルーティングプロトコルである.
RIPは,まずUDPのブロードキャストデータパケットを利用して,ルーティングテーブルを隣接するルータにブロードキャストする.この中には「メトリック」という,宛先ネットワークまでの距離(つまりルータのホップ数)を表す情報が含まれており,ルータはこの情報を利用して自身の保持するテーブルを最適化する.RIPでは,メトリックが最小となる経路が最適経路として採用される.また,メトリックの上限値は15となっており,これを超えた場合には到達不可能と判断される.

RIPは動的にネットワークの経路を変化させるため,ネットワーク管理の手間が軽減されるといったメリットがある一方,大規模なネットワークに適用すると最適経路が収束するまでに非常に時間がかかるといったデメリットがある.また,設定ミスや機器障害などが発生すると,誤った経路情報が広まってしまい,通信不能に陥ってしまうといったリスクもある.

\subsection{quaggaについて}
quaggaは,石黒邦宏氏が開発したルーティングソフトウェア GNU zebra の派生ソフトである.多くのUNIX系OSで動作し,OSPFv2, OSPFv3, RIPv1, RIPv2, RIPng, BGP-4 などのルーティングプロトコルを提供する.quagga をインストールすることによって,PCをソフトウェアルータとして動作させることが可能になる.
quaggaのコマンドラインは,業界最大手であるCiscoのルータに非常によく似ているため,Ciscoルータの練習台としてもよく利用される.興味のある人はCisco主催のCCNAという資格にチャレンジしてみよう.

\subsection{動的経路制御の実験}
\vspace{-6mm}
\subsubsection{NICの設定}
動的経路制御実験を行うにあたって,ルータのネットワーク情報を再設定する必要がある.ここでは,まずはルータの保持している経路情報を初期化してから以下のスクリプトを参考にNICの設定ファイルを編集する.

\begin{shadebox}
\begin{verbatim}
#### /etc/network/interfaces の編集####
sudo nano /etc/network/interfaces

auto eth0
iface eth0 inet manual
address 192.168.10.5/24
network 192.168.10.0
netmask 255.255.255.0


####ファイルを編集したら端末を再起動####
sudo shutdown -r now

\end{verbatim}
\end{shadebox}


\vspace{-6mm}
\subsubsection{quaggaの起動準備}
quaggaは,/etc/quagga に設定ファイルを設置することによって使用できるようになっており,デフォルトのままでは利用できない.
ここでは,quaggaを起動させるための必要ファイルを設置する.
なお,本実験ではquagga 1.2.4-1 を対象としている.quaggaのバージョンによってはサンプルファイルの場所が異なるため,注意すること.
\begin{shadebox}
\begin{verbatim}
####quaggaコアデーモンと周辺サンプルファイルのコピー####
sudo cp /usr/share/doc/quagga-core/examples/* ./

####必要なファイルを有効化する####
sudo cp zebra.conf.sample zebra.conf
sudo cp vtysh.conf.sample vtysh.conf
sudo cp ripd.conf.sample ripd.conf

\end{verbatim}
\end{shadebox}


\vspace{-6mm}
\subsubsection{quaggaの起動と設定}
ここからは,実際にquaggaを起動し,動的ルーティングプロトコルであるRIPを実装していく.quaggaはあくまでただのソフトだが,設定する際にはquaggaという名前の実在のルータ製品にコンソールケーブルを挿してログインするイメージを持つとやりやすい(まぁあくまで仮想的なものですが).

\begin{shadebox}
\begin{verbatim}
####quaggaの起動####
sudo service zebra start
sudo chkconfig zebra on

####RIPデーモンの起動####
sudo service ripd start
sudo chkconfig ripd on

########ここからルータ(quagga)本体にログインして設定する########

####対話モードでquaggaにログイン####
vtysh

####ルータを設定モードに切り替える####
configure terminal

####RIPの経路情報を認証無しで受信できるようにNICに設定する####
interface eth0
no ip rip authentication mode
exit

####RIPで扱う経路情報の通信路(ネットワーク)を登録する####
router rip
network 10.0.0.0/24
network 10.0.10.0/24
exit

####コンフィグを反映させる####
write memory

####ルーティングテーブルを確認する####
show ip route

\end{verbatim}
\end{shadebox}

上記設定を他のルータにも施す.

\vspace{-6mm}
\subsubsection{疎通確認}
pingコマンドを使ってホスト間の疎通確認をする.

\newpage

