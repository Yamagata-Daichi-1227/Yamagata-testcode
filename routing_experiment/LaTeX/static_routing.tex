%--------------------------------------------------------------
%   静的経路制御の実験
%--------------------------------------------------------------
\section{静的経路制御の実験}
様々な経路制御表を手動で設定することでルーティングを理解する.

\subsection{仮想マシンへのログイン}
本実験は,host1,host2,router1,router2 の計4台の仮想マシンを使って行う.
まず,各仮想マシンへsshで遠隔ログインする.
以降の実験は全て手元のコンソール画面にて行う.
\begin{shadebox}
\begin{verbatim}
####仮想マシンへの遠隔ログイン####
ssh host1@192.168.xx.xx

\end{verbatim}
\end{shadebox}
\vspace{4mm}

\subsection{ネットワーク情報の設定}
ネットワークの構築を行うにあたって,まずはNIC(Network Interface Card)の設定を行う.
UNIXでは,あらゆるインターフェースやデバイスをファイルとして扱うという特徴があり,
Debian系のOSでは,/etc/network/interfacesがNICの設定ファイルとして非常に有名である.
このファイルをnanoエディタで編集し,ネットワーク情報を書き込む.
\begin{shadebox}
\begin{verbatim}
####まずは状況確認をしよう####
ifconfig

####/etc/network/interfacesの編集####
sudo nano /etc/network/interfaces

####指定デバイスにネットワーク情報を書き込む####
auto eth0
iface eth0 inet static
address=192.168.xx.xx
network=192.168.xx.0/24
netmask=255.255.255.0

####端末の再起動####
sudo shutdown -r now

####設定が反映されていることを確認する####
ifconfig

\end{verbatim}
\end{shadebox}
\vspace{4mm}

\subsection{経路制御表を設定する}
ここからは,先程設定したホスト間における通信を行う.

IPネットワークには以下の基本が存在する.すなわち,
\medskip
\begin{shadebox}
  ネットワークの基本
\begin{itemize}
	\item 同一ネットワークに所属する端末は互いに通信可能である.
	\item 所属ネットワーク外に宛てた通信は,ゲートウェイに問い合わせないと届かない.
\end{itemize}
\end{shadebox}



\vspace{2mm}
host1 と host2 の間を通信する経路を以下に示す.

\medskip
\begin{shadebox}
\begin{verbatim}
####経路の例(行き帰りが同じ経路の場合)####
行き: host1→router1→router2→host2
帰り: host2→router2→router1→host1
\end{verbatim}
\end{shadebox}

この経路を「route」というコマンドを使って構築する.
パケットを目的の端末に届けるには,以下に上記の鉄則に従っているかがポイントとなる.

以下に示すサンプルを参考にして,経路テーブルを構築せよ.

\medskip
\begin{shadebox}
\begin{verbatim}
####経路制御表を表示する####
route 

####経路の追加の例####
route add -net 192.168.2.0/24 gw 192.169.4.1
route add default gw 192.169.4.1

####経路の消去の例####
route del -net 192.168.2.0/24 gw 192.169.4.1

\end{verbatim}
\end{shadebox}
\vspace{4mm}

\subsubsection{フォワーディングの有効化}
ルータの重要な機能の一つとして,フォワーディングというものがある.
これは,ルータがホストから受信したデータを,指定された別のマシンへ一切手を加えずに送信する機能のことで,ルータはこの「パケットのたらい回し機能」とルーティングテーブルを駆使することでルーティングを実現している.
この機能は,通常のマシンでは使用する必要が無い上に,全てのマシンにこの機能があると攻撃などに悪用される恐れがあるため,デフォルトでは無効化されている.
したがって,Linuxマシンをルータとして利用するためには,このフォワーディング機能を有効化する必要がある.

\begin{shadebox}
\begin{verbatim}
####sysctlのコンフィグファイルの編集####
sudo nano /etc/sysctl.conf

####フォワーディングの有効化####
net.ipv4.ip_forward=1

####コンフィグの反映####
sudo sysctl -p

\end{verbatim}
\end{shadebox}
