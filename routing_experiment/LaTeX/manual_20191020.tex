%--------------------------------------------------------------
%   電子情報実験C・D
%   「経路制御と防火壁」の実験テキスト
%--------------------------------------------------------------
\documentclass[10pt,a4paper]{jarticle}
\usepackage{ascmac}
\usepackage{graphicx}
\usepackage{exp}
\usepackage{ulem}
\usepackage{comment}
\pagestyle{empty}

%--------------------------------------------------------------
%   newcommand
%--------------------------------------------------------------
% desplay figures #1=width, #2=height, #3=filename, #4=caption, #5=label
\newcommand{\fig}[5]{
\begin{figure}[htbp]
\begin{center}
\resizebox{#1}{#2}{\includegraphics{#3}}
\end{center}
\caption{#4}
\label{#5}
\end{figure}
}

%--------------------------------------------------------------
\begin{document}
%--------------------------------------------------------------
%   タイトル
%--------------------------------------------------------------
\def\Title{初めての経路制御}
\begin{flushright}{\Huge\bf\Title}\end{flushright}

%--------------------------------------------------------------
%   実験目的
%--------------------------------------------------------------
\section{実験目的}
通信ネットワークの仕組みを,ルーティング(経路制御)によるネットワーク構築の実験を通して理解する.

%--------------------------------------------------------------
%   経路制御
%--------------------------------------------------------------
\section{経路制御} 
経路制御というのは,標識と同じである.

\medskip
\begin{center}
"目的地へ行くためには,ともかくこっちの道を通って行くこと."
\end{center}
\medskip

しばらく歩いて行くとまた標識かあり,"目的地へはこちらへ"と導かれる.それを繰り返すことでいずれは目的地に着く.
各標識に書かれる経路情報は,次の標識への案内だけで十分という点に注目すべきである.

例えば,ある町に行くのにバスを乗り継いで行きたい.
"○○方面はこのバス"という案内でバスに乗って行く.
終点についたら,そこでまたバス停の案内を見て,
また○○方面のバスはこれかと乗り継いで行くのである.
結果,目的とした町に到着することが可能である.
最初のバス停で,全ての乗り継ぎのバスがわかるのではない.
もし目的地までの全ての乗り継ぎ情報がわかるようにしようと思ったら,バス停の案内板には膨大な情報を書かなくてはならなくなるだろう.
目的地に行くために各バス停で得る情報は「とりあえず次に乗るべきバス」だけで十分である.

この例をコンピュータのネットワーク上で考えると,最初のバス停,即ち出発点は自ホスト,次のバス停は異なるネットワークの間にあるゲートウェイにあたる.
ゲートウェイはホストまたはルータである.
ネットワークが5つあるとしよう.
すると自ホストのバス停の案内板には,自ホストのネットワークを含めて5つの経路情報が書かれることになる.
社内のネットワークはこれで良い.
しかし,インターネットに接続すると,ネットワークの数は無数に存在する.
この経路情報を全て保持する事は不可能でありナンセンスである.
こういう場合,経路情報を保持しているプロバイダーに一任される.
このときの行き先経路をデフォルト経路という.

このようにTCP/IPネットワークの世界では小規模なネットワークから,インターネットのような大規模なネットワークでも対応できるように設計されている.
社内などの小規模ネットワークでは,経路情報を手動で設定することもある.
これを静的経路制御(スタティックルーティング)という.
手動ではなくて自動で経路情報を作成するやり方もある.
こちらは動的経路制御(ダイナミックルーティング)という.
社内ネットワークの経路制御を簡単にするには,
動的経路制御を内部ネットワークで使い,
インターネットへはデフォルト経路を設定する事である.

\subsection{静的経路制御(static routing)} 
\vspace{-6mm}
\subsubsection{利点}
静的経路制御には,動的経路制御に比べて明らかに優れている点がいくつかある.
最大の利点は予測可能な点である.
ネットワーク管理者が予め経路制御表を作成するため,
パケットがたどる2地点間の経路を常に正確に知り,かつ制御することができる.
動的経路制御では,どのホストおよびリンクが稼働しているか,
ルータが他のルータからの更新情報をどのように解釈するかによって,パケットがたどる経路が変わってくる.
また,静的経路制御は動的経路制御プロトコルを必要としないため,ルータやネットワークに余分なオーバーヘッドがかかることがない.
 
\vspace{-6mm}
\subsubsection{欠点}
静的経路制御は確に動的経路制御より有利だが,欠点がないわけではない.
静的経路制御は単純である代わりにスケーラビリティがない.
3つのルータに接続された5つのネットワークセグメントであれば,
各ルータからすべての宛先への最適経路を計算するのは難しくない.
しかし,大半のネットワークはこれよりずっと大きい.
十数個を越えるルータで相互に接続された200のセグメントからなるネットワークだと,どのようなルーティングになるだろうか.
%どのようになるか書くべきでは?(羽生

\subsection{動的経路制御(dynamic routing)}
\vspace{-6mm}
\subsubsection{利点}
動的経路制御が静的経路制御に比べて優れているのは,
スケーラビリティと適応性である.
動的経路制御されるネットワークは,早くかつ大きく成長できる.
また,このようなネットワークの成長,あるいは,1つ以上のネットワークコンポーネントの故障によってもたらされるネットワークトポロジの変化にも対応できる.

動的経路制御プロトコルでは,ルータは他のルータとやり取りする事でネットワークトポロジについて学習する.
各ルータは,その存在と自分が提供できる経路をネットワーク場の他のルータに対してアナウンスする.
したがって,新しいルータを追加したり,既在のルータに新しいセグメントを接続したりすると,他のルータはその追加に付いて知り,
それに会わせて自身の経路制御表を修正する.
管理者がルータをいちいち再設計をして,
ネットワークが変更されたことをルータに知らせる必要はない.
同様にネットワークを移動した場合も,
他のルータはその変更について知らされる.
管理者は,移動されたセグメントに接続しているルータの設定を変更するだけでよい.
これによりエラーが起こりにくくなる.

\vspace{-6mm}
\subsubsection{欠点}
動的経路制御の最大の欠点は,
静的経路制御に比べて複雑さが大きい点だ.
ネットワークトポロジに関する情報をやり取りするのは,
「こっちが到達できる宛先は○○だ」などと言い合うような単純な話ではない.
動的経路制御プロトコルに参加するルータは,
他のルータに送信する情報を正確に決定しなければならない.
それだけではなく,自分が他のルータから学習した複数経路の中から,宛先に到達するための最良の経路を選択する必要がある.
その上,ネットワークの変化に対応しようとするなら,古くて使えなくなった情報を経路制御表から削除できなければならない.
どれが古くて使えない情報か判断するためのロジックは,
ルーティングプロトコルをますます複雑にする.
しかし念ながら,ネットワークないのさまざまな状況に対処できるプロトコルほど,その分,複雑になりやすい.
複雑になると,プロトコルの実装エラーが発生率が増加する.

%--------------------------------------------------------------
%   実験で使うコマンド
%--------------------------------------------------------------
\section{実験で使うコマンド}

\subsection{ifconfig}
ネットワークインタフェースの情報を表示するコマンドである.
引数無しで実行するとそのマシンの全てのネットワークインタフェースの情報を表示する.
IPアドレス,ネットマスク,Macアドレス,インタフェースの状態などを知ることができる.

\subsection{route}
ルーティングテーブル(経路制御表)を確認したり,設定するコマンドである.
Destinationはパケットの宛て先のIPアドレスまたはIPアドレスが所属するネットワークアドレスを示す.
Destinationが"default"となっている行は,
パケットの宛て先が他のDestinationに載っていない場合に適用される.
Gatewayはパケットを次に転送すべきルータのIPアドレスを示す.

\subsection{ifup/ifdown}
ネットワークインターフェースを有効化・無効化する.

\subsection{sysctl}
Linuxカーネルパラメータの参照・変更を行う.

\subsection{ssh}
SSHプロトコルによって他のマシンと暗号化通信を行うコマンドである.
リモート通信には従来telnetが使われていたが,
telnetは暗号化をしない通信でありセキュリティ上よくないため,
現在はリモート通信にはsshを使うのが一般的である.

\subsection{ping}
pingコマンドは,「あるホストに接続できるかどうか知りたい」,
「あるホストが動いているかどうか知りたい」というときに確認できるものである.
その仕組みは,「ある大きさのデータを指定された相手のホストに送り,
相手のホストからの返事を待つもの」であり,
プロトコルとしてはICMPが用いられている.
このプロトコルは,インターネット世界でのエラー内容通知や,
制御用のメッセージを送るためのものであり,
pingコマンドの場合には
echo(エコー要求)というタイプのICMPメッセージを送信している.
メッセージを受け取ったホストは
echo reply(エコー応答)というメッセージを送り返してくるので,
これが戻ってくれば自分のホストと相手のホスト間のネットワーク
および相手のホストが正常に動作していることになる.

\vspace{-6mm}
\subsubsection{pingコマンドの出力結果について}
\begin{screen}
\begin{verbatim}
PING gold.router (192.168.7.11): 56 data bytes
64 bytes from 192.168.7.11: icmp_seq=0 ttl=255 time=0.196 ms	
64 bytes from 192.168.7.11: icmp_seq=1 ttl=255 time=0.148 ms
64 bytes from 192.168.7.11: icmp_seq=2 ttl=255 time=0.110 ms
64 bytes from 192.168.7.11: icmp_seq=3 ttl=255 time=0.115 ms
\end{verbatim}
\end{screen}

1行目は,goldへ54bytesのデータを送信する.
2行目からはecho replyを受け取った結果だが,
pingコマンドはデフォルトで毎秒1回のペースでICMP echoパケットを送信するため,
コマンドを実行すると,1行づつ時間をおいて表示される.
行頭の数値は,受信したパケットの大きさである.
シーケンス番号はecho requestを送信するときに
0から順に1づつカウントアップされていくわけだが,途中の番号が欠ける場合もある.
これは,行きか帰りかはわからないが,途中のパケットが失われたことを意味していて,
ネットワークの状態が良くない場合に起こる現象といえる.
ttl(Time To Live)は,パケットがどれだけネットワーク間を飛び続けられるかを示している.
0になった時点でパケットは破棄される.
最近のシステムは初期値を255にしてあるものが多いので,
TTLが244ということは,相手のホストを出発して自分のところに到達するまでの間,
11台のルーターを経由したと考えられる.
RTT(Round Trip Time)は,
パケットが相手のホストまで行って再び帰ってくるまでの往復時間を示している.
片道の時間ではないことに注目して欲しい.
往復時間を半分にすれば片道かというと,
行きと帰りでは通る道筋が異なっている場合もある.

pingコマンドは放っておくと動き続けるので,
適当なところでCtrl+cキーを入力して動きを止める.
すると最後に次のような統計情報が表示される.
\begin{screen}
\begin{verbatim}
---- gold.router ping statistics ---
100 packets transmitted, 100 packets received, 0% packet loss
round-trip min/avg/max/stddev = 0.105/0.141/0.199/0.020 ms
\end{verbatim}
\end{screen}
送出したecho requestパケットの数と戻ってきたecho replyパケットの数,
失った(戻ってこなかった)パケットの割合,RTTの最小値や平均値,最大値,
標準偏差が表示される.

\vspace{-6mm}
\subsubsection{pingコマンドによるネットワーク状況の判断}
出力結果より,まず,icmp\_seqの値に着目する.
たまに番号が1つ飛ぶ程度であればほとんど問題ないが,
シーケンス番号がひどく飛ぶようであれば,
ネットワークの状態はかなり悪いと考えられる.
そのようなケースでは,
相手のホストをリモートから利用しても反応がないなどの症状が発生する.
次にRTTに着目する.
RTTは小さい値の方が良いわけだが,
最小値は回線速度や経由するルーター数である程度決まってしまう.
したがって,ばらつき加減を見た方が良い.
最後の統計情報に注目して,
最小値と最大値が大きく離れている場合には,
混雑している可能性が高い.

\subsection{traceroute}
トレースルート(traceroute)は,データグラムが目的地まで経由するルータ,
すなわち経路を知るためのアプリケーションであり,
ICMP time exceeded message を利用する.
Time exeeded message はデータグラムの寿命であるTTLが0となった時に
ルータパケットが廃棄されたことを発信源に知らせるメッセージである.

\subsection{tcpdump}
ネットワーク上を流れるパケットを監視するコマンドである.
実行すると,指定したネットワークインタフェース上を通過する
全てのパケットの情報を画面に表示し続ける.
不要なパケットの情報を表示しないように条件を付けることもできる.

\subsection{quagga}
クアッガ(quagga)コマンドは,RIPに対応した動的経路制御コマンドである.
実行することで動的に経路制御表が作成される.

\subsection{nano}
nanoエディタを呼び出す.


%--------------------------------------------------------------
%   覚えておいたほうが良いUNIXの基本コマンド
%--------------------------------------------------------------
\section{覚えておいたほうが良いUNIXの基本コマンド}

\subsection{cd}
指定したディレクトリに移動する.

\subsection{ls}
カレントディレクトリにあるファイルおよびディレクトリの一覧を表示する.

\subsection{less}
ファイルの中身を閲覧する.

\subsection{cat}
ファイルの中身を標準出力(画面のこと)に出力する.

\subsection{mv}
ファイルを指定の場所に移動する.あるいは,指定したファイルの名前を変更する.

\subsection{pwd}
カレントディレクトリの絶対パス(つまり今自分がいる場所)を表示する.

\subsection{sudo}
後に続くコマンドを管理者として実行する.

%--------------------------------------------------------------
%   nanoの使い方
%--------------------------------------------------------------

\section{nanoの使い方}
nanoはlinuxに標準でインストールされているテキストエディタである.
他に有名なものとしてはemacsやviなどがある.
nanoは扱いに若干の癖があるため,ここに基本操作方法を示す.

\begin{shadebox}
\begin{verbatim}
####エディタの起動####
nano hoge.txt

####文書の保存####
Ctrl+w
-----------------------------------------------
| File Name to Write: hoge.txt  <Enterを押す> |
-----------------------------------------------

####エディタの終了####
Ctrl+x

# 保存する前に終了しようとすると,
-----------------------------------------------------------------
| Save modiflier buffer (ANSWERING "No" WILL DESTROY CHANGES) ? |
| Y Yes                                                         |
| N No     ^C cancel                                            |
-----------------------------------------------------------------
# というメッセージが表示される.保存しないで終了する時はN(またはn)を入力する.

\end{verbatim}
\end{shadebox}
\vspace{4mm}

\newpage

%%%%%%%%%%%%%ここに静的経路制御の実験を書く.
%--------------------------------------------------------------
%   静的経路制御の実験
%--------------------------------------------------------------
\section{静的経路制御の実験}
様々な経路制御表を手動で設定することでルーティングを理解する.

\subsection{仮想マシンへのログイン}
本実験は,host1,host2,router1,router2 の計4台の仮想マシンを使って行う.
まず,各仮想マシンへsshで遠隔ログインする.
以降の実験は全て手元のコンソール画面にて行う.
\begin{shadebox}
\begin{verbatim}
####仮想マシンへの遠隔ログイン####
ssh host1@192.168.xx.xx

\end{verbatim}
\end{shadebox}
\vspace{4mm}

\subsection{ネットワーク情報の設定}
ネットワークの構築を行うにあたって,まずはNIC(Network Interface Card)の設定を行う.
UNIXでは,あらゆるインターフェースやデバイスをファイルとして扱うという特徴があり,
Debian系のOSでは,/etc/network/interfacesがNICの設定ファイルとして非常に有名である.
このファイルをnanoエディタで編集し,ネットワーク情報を書き込む.
\begin{shadebox}
\begin{verbatim}
####まずは状況確認をしよう####
ifconfig

####/etc/network/interfacesの編集####
sudo nano /etc/network/interfaces

####指定デバイスにネットワーク情報を書き込む####
auto eth0
iface eth0 inet static
address=192.168.xx.xx
network=192.168.xx.0/24
netmask=255.255.255.0

####端末の再起動####
sudo shutdown -r now

####設定が反映されていることを確認する####
ifconfig

\end{verbatim}
\end{shadebox}
\vspace{4mm}

\subsection{経路制御表を設定する}
ここからは,先程設定したホスト間における通信を行う.

IPネットワークには以下の基本が存在する.すなわち,
\medskip
\begin{shadebox}
  ネットワークの基本
\begin{itemize}
	\item 同一ネットワークに所属する端末は互いに通信可能である.
	\item 所属ネットワーク外に宛てた通信は,ゲートウェイに問い合わせないと届かない.
\end{itemize}
\end{shadebox}



\vspace{2mm}
host1 と host2 の間を通信する経路を以下に示す.

\medskip
\begin{shadebox}
\begin{verbatim}
####経路の例(行き帰りが同じ経路の場合)####
行き: host1→router1→router2→host2
帰り: host2→router2→router1→host1
\end{verbatim}
\end{shadebox}

この経路を「route」というコマンドを使って構築する.
パケットを目的の端末に届けるには,以下に上記の鉄則に従っているかがポイントとなる.

以下に示すサンプルを参考にして,経路テーブルを構築せよ.

\medskip
\begin{shadebox}
\begin{verbatim}
####経路制御表を表示する####
route 

####経路の追加の例####
route add -net 192.168.2.0/24 gw 192.169.4.1
route add default gw 192.169.4.1

####経路の消去の例####
route del -net 192.168.2.0/24 gw 192.169.4.1

\end{verbatim}
\end{shadebox}
\vspace{4mm}

\subsubsection{フォワーディングの有効化}
ルータの重要な機能の一つとして,フォワーディングというものがある.
これは,ルータがホストから受信したデータを,指定された別のマシンへ一切手を加えずに送信する機能のことで,ルータはこの「パケットのたらい回し機能」とルーティングテーブルを駆使することでルーティングを実現している.
この機能は,通常のマシンでは使用する必要が無い上に,全てのマシンにこの機能があると攻撃などに悪用される恐れがあるため,デフォルトでは無効化されている.
したがって,Linuxマシンをルータとして利用するためには,このフォワーディング機能を有効化する必要がある.

\begin{shadebox}
\begin{verbatim}
####sysctlのコンフィグファイルの編集####
sudo nano /etc/sysctl.conf

####フォワーディングの有効化####
net.ipv4.ip_forward=1

####コンフィグの反映####
sudo sysctl -p

\end{verbatim}
\end{shadebox}

% \newpage
%%--------------------------------------------------------------
%   動的経路制御の実験
%--------------------------------------------------------------

%
% 実験の動機
% RIPとは
% quaggaとは
% 実験開始
%  - NICの設定をする
%  - そもそものquaggaを使えるようにする
%  - quaggaの起動とripの起動
%  - quaggaの設定
%  - 経路表の確認と疎通確認

\section{動的経路制御の実験}
%あらまし
静的経路制御では,各端末のNICの設定を行った後,routeコマンドを用いて手動で経路制御表の作成を行った.
しかしながら,この方法ではネットワーク構成の一部を変化させる際に大幅な改変を行うことになる.
そこで,本実験では自動的にルーティングテーブルを更新するプロコトルとして有名なRIPによる動的経路制御を行う.

\subsection{RIPプロトコル}
本実験で使用するルーティングプロトコルであるRIPについて簡単に説明する.

RIP(Routing Information Protocol)は,UDP/IP通信用のダイナミックルーティングプロトコルである.
RIPは,まずUDPのブロードキャストデータパケットを利用して,ルーティングテーブルを隣接するルータにブロードキャストする.この中には「メトリック」という,宛先ネットワークまでの距離(つまりルータのホップ数)を表す情報が含まれており,ルータはこの情報を利用して自身の保持するテーブルを最適化する.RIPでは,メトリックが最小となる経路が最適経路として採用される.また,メトリックの上限値は15となっており,これを超えた場合には到達不可能と判断される.

RIPは動的にネットワークの経路を変化させるため,ネットワーク管理の手間が軽減されるといったメリットがある一方,大規模なネットワークに適用すると最適経路が収束するまでに非常に時間がかかるといったデメリットがある.また,設定ミスや機器障害などが発生すると,誤った経路情報が広まってしまい,通信不能に陥ってしまうといったリスクもある.

\subsection{quaggaについて}
quaggaは,石黒邦宏氏が開発したルーティングソフトウェア GNU zebra の派生ソフトである.多くのUNIX系OSで動作し,OSPFv2, OSPFv3, RIPv1, RIPv2, RIPng, BGP-4 などのルーティングプロトコルを提供する.quagga をインストールすることによって,PCをソフトウェアルータとして動作させることが可能になる.
quaggaのコマンドラインは,業界最大手であるCiscoのルータに非常によく似ているため,Ciscoルータの練習台としてもよく利用される.興味のある人はCisco主催のCCNAという資格にチャレンジしてみよう.

\subsection{動的経路制御の実験}
\vspace{-6mm}
\subsubsection{NICの設定}
動的経路制御実験を行うにあたって,ルータのネットワーク情報を再設定する必要がある.ここでは,まずはルータの保持している経路情報を初期化してから以下のスクリプトを参考にNICの設定ファイルを編集する.

\begin{shadebox}
\begin{verbatim}
#### /etc/network/interfaces の編集####
sudo nano /etc/network/interfaces

auto eth0
iface eth0 inet manual
address 192.168.10.5/24
network 192.168.10.0
netmask 255.255.255.0


####ファイルを編集したら端末を再起動####
sudo shutdown -r now

\end{verbatim}
\end{shadebox}


\vspace{-6mm}
\subsubsection{quaggaの起動準備}
quaggaは,/etc/quagga に設定ファイルを設置することによって使用できるようになっており,デフォルトのままでは利用できない.
ここでは,quaggaを起動させるための必要ファイルを設置する.
なお,本実験ではquagga 1.2.4-1 を対象としている.quaggaのバージョンによってはサンプルファイルの場所が異なるため,注意すること.
\begin{shadebox}
\begin{verbatim}
####quaggaコアデーモンと周辺サンプルファイルのコピー####
sudo cp /usr/share/doc/quagga-core/examples/* ./

####必要なファイルを有効化する####
sudo cp zebra.conf.sample zebra.conf
sudo cp vtysh.conf.sample vtysh.conf
sudo cp ripd.conf.sample ripd.conf

\end{verbatim}
\end{shadebox}


\vspace{-6mm}
\subsubsection{quaggaの起動と設定}
ここからは,実際にquaggaを起動し,動的ルーティングプロトコルであるRIPを実装していく.quaggaはあくまでただのソフトだが,設定する際にはquaggaという名前の実在のルータ製品にコンソールケーブルを挿してログインするイメージを持つとやりやすい(まぁあくまで仮想的なものですが).

\begin{shadebox}
\begin{verbatim}
####quaggaの起動####
sudo service zebra start
sudo chkconfig zebra on

####RIPデーモンの起動####
sudo service ripd start
sudo chkconfig ripd on

########ここからルータ(quagga)本体にログインして設定する########

####対話モードでquaggaにログイン####
vtysh

####ルータを設定モードに切り替える####
configure terminal

####RIPの経路情報を認証無しで受信できるようにNICに設定する####
interface eth0
no ip rip authentication mode
exit

####RIPで扱う経路情報の通信路(ネットワーク)を登録する####
router rip
network 10.0.0.0/24
network 10.0.10.0/24
exit

####コンフィグを反映させる####
write memory

####ルーティングテーブルを確認する####
show ip route

\end{verbatim}
\end{shadebox}

上記設定を他のルータにも施す.

\vspace{-6mm}
\subsubsection{疎通確認}
pingコマンドを使ってホスト間の疎通確認をする.

\newpage



%------------------------------------------------------------------------------

\end{document}
 
